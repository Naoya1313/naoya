\chapter{はじめに}
\thispagestyle{myheadings}

% **DIOCOMO2019 ========================================================**
\section{背景および研究目的}
近年コロナ禍の影響からNetflixやYouTubeといった家で楽しむコンテンツの普及が進んでいる.しかし現在のコンテンツでは家で映画館のような3D映像や4DXなどの映像体験を得るのが難しい.
現在他者との共有ツールが実装されているコンテンツとして,ニコニコ動画の画面にリアルタイムで視聴者のコメントを流し,他者との共有をしている.YouTubeでは”最もウケた”場面が分かる拡張機能が提供されている.機能として動画のシーク中,シークバーの上に表示されるグラフで,最も再生された場所が表示されるようになっている.しかし現在の機能では本当の感情を表現しづらい.



% **DIOCOMO2020 (ryoga)====================================================**
\section{エフェクト重畳}
コンテンツ視聴時の生体データを取得し、そのデータを基にコンテンツにエフェクトを重畳する.それによりコンテンツにさらなる楽しみ方を加えるのを目的とする.ニコニコ動画では画面にリアルタイムで視聴者のコメントを流し、他者との共有をしている.
本研究では生体データを使用しコンテンツのどの部分が盛り上がっているか他者との共有をする.生体データを使用しコンテンツに新しい効果が得られるのではないかと考えた.
\label{sec:example}


\section{論文構成}
