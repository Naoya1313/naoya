\thispagestyle{myheadings}
\chapter{関連研究}

% DICOMO2019 ########################################################################################################
\section{モノの状態推定に関する関連研究}
室内にあるモノの状態推定には,加速度センサや振動センサなどを利用する方法やWi-Fiの電波を利用する方法など様々な手法が提案されてきた.
前川ら\cite{TagAndThink}は様々なセンサを搭載したセンサノードをモノに取り付け,そこから得られるセンサ情報と事前に用意しておいたそのモノ固有の状態遷移図を比較し,自動でモノの状態と何に付けられているのかを推定している.
角速度や照度などから,センサノードが取り付けられている状況や状態変化を検出するという手法である.
この手法ではセンサノードがどんなモノに付いていて,どんな状態変化をしたか推定が可能である.

消費電力の変化から電気機器の状態推定を行っている研究がいくつかある.
例えば,機器やコンセントごとの消費電力を計測できる細粒度電力センサを使用し,電気機器の電力消費の変化から浪費電力の検出・分類が行われている\cite{sairyu}.
その他にも電気機器の運転モードの切り替えや開閉などの状態変化によって起こる消費電力の変化から状態の推定を行い,そこから得られる時系列の情報から人物の位置推定が行われている\cite{energy}.
これら二つの手法では,電気機器の状態変化から起こる消費電力の増減に着目し推定を行っているため,電化製品に対しては有効な手法ある.
一方で,電気を用いない家具や雑貨に対しては適用が不可能である.

Wi-Fi電波を用いて,室内状況や人の状態を推定する研究がある.
例えば室内状況の推定では,Wi-Fiのチャンネル状態情報を用いてドアの開閉検知\cite{WifiChannel}や通過検知\cite{LANgate},会議室の状態検出\cite{Room-State}が行われている.
また,人の状態推定では呼吸の検出\cite{Human-Respiration}や失神検知\cite{WiFi-Toilet}が行われている.
これらの研究は,Wi-Fi電波を利用するという理由から推定対象物にセンサを取り付ける必要が無く,導入が簡単に行える.
しかし,電波の状態を推定の指標とするため,小さいモノや頻繁に移動を伴うモノでは状態変化の推定が難しい.

そこで,これらの問題の解決策として本研究ではBLEビーコンを用いる.
BLEビーコンは,低消費電力であるため長期間の稼働が可能であり,電池交換といった運用コストを小さくできる.
また,小型・軽量という特徴から設置が容易で,直接モノの内部への設置も可能である.
このような特徴から,マーケティング\cite{bleUse}から在室検知・位置推定まで幅広く活用されている.
在室検知では,BLEビーコン電波のRSSI値を用いて在室を推定する研究が行われている\cite{home-location, stay-estimation, en-AreaUsed, Finding_by_Counting, dakoku_system, makoto, konzatsu}.
また,距離や障害物の有無によって変化する電波強度をもとに,位置や状態を推定する研究が数多く行われている\cite{move-tracking, BLE-Localization, IoMT, tandem, blespot, en-door}.
本研究では,これらの研究で利用されているBLEビーコンの特徴に着目し,日常の生活空間内におけるモノの状態変化の推定を行う.


% DICOMO2020(ryoga) ########################################################################################################
\section{睡眠位置認識に関する関連研究}
睡眠位置認識に関する研究は様々行われている.
ベッドの脚それぞれに荷重センサを取り付け位置を認識する研究\cite{ML}やサービスがあり,高齢化による需要の増加や介護業界の人手不足,介護人の負担増加といった問題を解決するために支援を行っている.
例えばリコーの見守りベッドセンサシステム\cite{riko}では,リアルタイムにベッド上での位置把握が可能であり,そこから活動量や離床のタイミングがわかる.
さらにそれらの情報は,ネットワークを介してナースステーションや家族の自宅から確認ができる.
しかし,導入には費用や手間がかかってしまうため在宅介護などで利用するにはハードルが高い.
またこの手法ではベッドの脚にセンサを取り付ける必要があるため,布団では適用が難しい.


布団にも適用できる睡眠位置認識では,圧力センサを用いた研究が多く行われている.
カメラを用いた動画像処理を行う研究\cite{Multimodal}もされているが,プライバシーや被撮影者負担などの観点から避けられる傾向にあり,代わりにセンサを用いた手法が多く提案されている.
西田らは221個の圧力センサを7cm間隔で並べた圧力分布測定シートを用いて,体位の測定に加え呼吸の測定を行っている\cite{atu}.
睡眠位置の判定には,圧力分布画像を処理し芯線を抽出する.
さらに芯線上の圧力最大地点の特徴から仰臥位と腹臥位を判別し,横隔膜の振幅の特徴と圧力分布を組み合わせて呼吸検出を行っている.


また近年ではシーツ型のセンサや,シーツや衣服といった布にセンサを織り込んだものが多く使用されている\cite{orimono, seat}.
岩瀬らは寝姿体圧画像から睡眠位置推定と関節位置推定を行っている\cite{kansetu}.
関節位置推定モデルに人物領域推定によるノイズ抑制と姿勢分類情報を組み込み推定を行う.% ←←←なんか変
また小野瀬らは衣類型のセンサを用いて圧力分布を計測しており,シーツ型センサとの比較を行っている\cite{irui_hikaku}.
これらの手法では,測定された圧力から特定の部位に負担がかかっていないか把握できるため,褥瘡対策に有効である.
しかし設置と使用には専用のセンサや装置が必要になるため,やはり一般の人では導入が難しい.





そこで我々は,BLEビーコンの受信電波強度を用いた睡眠位置認識手法を提案する.
BLEビーコンであれば家電量販店などで手軽に安価で入手でき,電波の受信もスマートフォンでできるため専用の装置を用意する必要が無い.
我々は以前,受信電波強度を用いた状態推定を提案しており,その手法をもとに位置認識を行う.


% DICOMO2020(ogane) ########################################################################################################
\section{車椅子の移動認識に関する関連研究}
車椅子の移動認識手法として,センサやGPS,カメラを用いた手法など様々な手法が提案されている.
%モーションセンサ
センサを用いた手法では,加速度センサや角速度センサ,ロータリエンコーダといった,モーションセンサを用いた手法が数多く提案されている\cite{en-Accelerometer, en-kalman}.
これらの手法は,車椅子に直接センサを取り付けてセンシングするため,正確な動きの情報が得られる.
しかし,この手法は測定データから得られる移動方向と距離をもとに,初期位置からの相対的な移動を推定していくものであるため,長距離移動した場合ズレが蓄積してしまう.
この問題に対処するため,長谷川らはBLEビーコンを用いて誤差を補正し,車椅子バスケにおける選手の位置推定を行う手法を提案している\cite{baske}.
車椅子に設置された受信機でBLEビーコンの電波を受信し,そのBLEビーコンIDと紐付けられた位置情報をもとに,位置誤差を修正するという手法である.
このようにモーションセンサを用いた手法では,位置誤差を修正する必要がある.

%GPS
位置誤差が蓄積しない手法として,GPSを用いた手法がある.
GPSは現在位置を緯度・経度の絶対座標として取得できるため,屋外において高精度な移動認識が実現できる.
この特徴を利用し,屋外の道路において左右どちらの歩道を通行したか推定する研究がある.
阿部らは,人のそばにある建物上空のGPS信号は受信し難いが,上が開けている車道側上空のGPS信号は受信されやすいという特徴に着目し,歩道単位での位置推定を実現している\cite{gps}.
この研究はGPS信号のみを使用して歩行通路判定を行うため,マップマッチングなどの補正をする必要がなく,実装が簡単に行えるメリットがある.
一方で,GPS信号は屋外でしか受信できないため,屋内での移動認識は不可能である.

%カメラ
屋内においても移動認識を実現する手法として,カメラを用いた手法がある.
この手法は,あらかじめ目印となる物とその座標を結びつけておき,カメラでその目印を検出して位置推定を行う.
この特徴を用いて車椅子ナビゲーション\cite{marker}や,軌道追跡\cite{en-baske}を実現する研究がある.
これらの研究は目印となる物を設置するだけで位置推定環境が整うため,専門知識がない人でも簡単に導入が行える.
しかしながら,目印の検出には画像処理が必要であり,計算量のコストが他の手法と比べて高いという問題がある.

%スマホセンサ
導入コストが低い移動認識手法として,スマートフォンに内蔵されたセンサを用いる手法がある.
ワッタナワラォンクンらは移動速度を測る車輪センサと,スマートフォンの方位センサを組み合わせて測位をしている\cite{navi}.
また,測位結果をもとに坂道や階段を回避した目的地への最適ルートを計算し,スマートフォン上に図面データと共に表示する.
この手法はスマートフォンをナビゲーションシステムとして使うだけでなく,測位のためのセンサとしても使用している.
そのため,モーションセンサだけを使用する手法と比べて導入コストを抑えられる.
岩崎らは室内の天井にスピーカを取り付け,そこから発せられる周波数の異なる音波をスマートフォンで受信し,測位する手法を提案している\cite{microphone}.
この手法は車椅子側にセンサを取り付ける必要が無いため,車椅子を多く使用する施設では導入コストの大幅な削減が実現できる.
一方で,スマートフォンは機種によってセンサ精度に違いがあるため,人によって得られる結果が異なる可能性がある.

%モーションセンサの応用例
通常,移動認識に使用される情報を他の用途に活用する研究がいくつか行われている.
まず,加速度センサの情報をもとに,車椅子使用者の消費カロリーを推定する研究がある.
谷本らは加速度の変化から車輪を漕いだ回数を推定し,その際の強度を3つのレベルに分類するという手法で車椅子使用時の消費カロリーを推定している\cite{en-calorie}.
この研究は加速度の情報から消費カロリーを算出しているため,移動認識と運動量推定が同時に実現できる.

また,センサデータに基づいたバリア情報の評価を目指す研究がある\cite{OperationModel, RoadEval, on-demand, en-Mahalanobis}.
通常,移動認識に使用される情報を活用して,歩道の傾斜や障害物といったバリア情報を発見・評価するという研究である.
これらの研究は,その評価したバリア情報から車椅子使用者の行動支援をできるだけでなく,障害物を目印として移動認識における位置誤差の修正にも応用が可能である.
