\thispagestyle{myheadings}
\chapter{関連研究}
本章では本研究に関連するコンテンツについての研究や生体データを用いた研究,
重畳提示手法の研究,体験拡張研究の4つのテーマについて紹介する.本研究と比較し関連性や新規性を述べる.



\section{情報媒体における研究}
コンテンツは大きく分けデジタルコンテンツとアナログコンテンツに分類される.
デジタルコンテンツには動画や電子書籍,アナログコンテンツには本や新聞紙などがある.
アナログコンテンツである本や新聞紙など紙で読む方がデジタルコンテンツで読むより記憶に残る研究がある\cite{books}.
スマートフォンの普及で本や新聞紙などを紙で読む機会が減った.そこで本研究のシステムを使えば,
アナログコンテンツにエフェクトを重畳し本や新聞紙を読む楽しさが増え,紙で読む機会も増えると考える.
例として本を読んでいる時は心拍数を使い,面白いと感じる記事を読んでいるときの心拍数を計測するなどコンテンツに合わせた生体データを取得する.
本への重畳提示は,徐々に部屋を暗くしたり雑音をはじめの方は流していて盛り上がるポイントが来たら無音にするなどで集中しやすい環境に近づけられる.


\section{生体情報から感情を推定する研究}

生体データを取得したときに得られる人の感情について言及する.
生体データには指紋や顔の表情など数多くの種類が存在する.
本研究では生体データを基にコンテンツにエフェクトを重畳するため,生体データからコンテンツに対して視聴者がとる反応を抜き出す必要がある.
生体データとして顔の表情から人の感情を読み取る研究がある\cite{hyoujou,hyoujou2}.
また瞳孔に注目し人が覚醒しているかを推定する研究や.人が居眠り運転しているのを推定する研究もある\cite{doukou}.
コンテンツ視聴時の心拍数や皮膚電気抵抗からコンテンツの質を測る研究\cite{situ}や,生体情報を分析し人の行動を予測する研究\cite{eigo}もある.
これらの研究より生体データから感情が読み取れる有効性が示されている.
図\ref{seitaidata}のようにさまざまな生体データを用いて人の楽しいや辛いなどといったいくつもの感情を推定できる.
そこで本研究ではコンテンツを視聴している時の生体データを取得し,生体データから人の感情を読み取りコンテンツを視聴している時のどの部分が盛り上がっているポイントかを推定する.
コンテンツに対し抱いた感情をエフェクトという形で重畳し,コンテンツに新たな楽しみ方を加える.

\begin{figure}[H]
    \centering
    \includegraphics[width=8cm]{images/chapter2/heart.png}
    \caption{生体データからさまざまな感情を取得\cite{kanjoutaiken}}
    \label{seitaidata}
\end{figure}


\section{生体情報を基にフィードバックする研究}

生体データを基にフィードバックする研究を分析した.
生体データを用いてユーザーに学習支援をする研究がある\cite{seitai1,seitai2,seitai3}.
生体データの使用によって,ユーザーの現状を正確に把握し適切に学習支援が可能である.
また,生体情報を用いて車の居眠り運転の防止する研究\cite{unten}や,体調を生体データによって管理し事前に病気を予防する研究\cite{kibun,yobou,yobou2}はいくつかある.
生体データを用いてユーザーの日々の行動を記録し,事前にユーザーへ通知を行い事故や病気を予防している.
生体情報を用いるIoT機器も数多く存在する.
生体データをIoT機器で収集しライフケア,ヘルスサービスへ活用している.
例として図\ref{iot}に示す.
運動している時や睡眠している時に生体データを収集し,その人にあった運動不足や疲労の回復度など適切なフィードバックが可能である.
これらの研究に対し我々の研究では生体データをコンテンツに適応する.
生体データを用いてユーザーの感情を分析し,コンテンツへ迫力の増加や恐怖の軽減などを目的とする.


\begin{figure}[H]
    \centering
    \includegraphics[width=15cm]{images/chapter2/body_img02.png}
    \caption{ウェアラブルデバイスを用いた生体情報のIoT\cite{nec}}
    \label{iot}
\end{figure}


\section{コンテンツにエフェクトを重畳して面白さを増幅する研究}
コンテンツにエフェクトを重畳して面白さを増幅する研究について述べる.
視線検出装置を用いてユーザがディスプレイ上の画像において任意の点に注目した際に,奥行きにフォーカスされた画像を映像に随時反映し奥行き感を強化する手法が提案されている\cite{shamo1,shamo2}.
しかし、提案システムはあらかじめ用意した最大256枚からなる画像を利用しており,手軽にコンテンツの視聴体験を拡張,増幅できるとは言い難いため\cite{shamo3},本研究はエフェクト重畳提示手法を使用した. 

\section{中心視と周辺視の持つ特性の研究}
中心視と周辺視の持つ特性の研究について述べる.
動画コンテンツの周辺にカメラ映像を配置し、動画コンテンツを見ている人の周辺視に何気なく現実世界を提示し,視聴中の動画コンテンツに連動し、
周辺に提示されているカメラ映像に対してエフェクトを付与してカメラ映像内の現実世界を変容させ動画視聴中のユーザの周辺視野を刺激し、動画コンテンツに対する迫力感や恐怖感などを増幅する研究がある\cite{shamo4}.
周辺視では複雑な文字などの図形を部分的にしか認識ができず、中心に近づくほど、より複雑な図形の知覚が可能であり\cite{shamo5},CFF(明滅する光のちらつき感がちょうど消失する周波数)で示される中心視と周辺視の感度差においては、
光のちらつきを最も敏感に感じ取るのは中心視ではなく周辺視であると判明している\cite{shamo6}.光のちらつきを周辺視がより敏感に感じられるという特性から,周辺視を刺激しコンテンツ視聴体験の拡張を目指し周辺視が人間の知覚能力に及ぼす影響について示している\cite{shamo7}.
周辺視野ほど明るいものがより明るく見え,暗いものはより暗く見えるという輝度対比効果について明らかにしているため \cite{shamo8},我々の研究では動画の周辺にエフェクトを提示し,コンテンツに新たな効果を加える.

\section{ユーザの体験を拡張する研究}     
ユーザの体験を拡張するという研究について述べる.FocusPlusContextDisplayは,高解像度のディスプレイを中心視野用に,低解像度のプロジェクタを周辺視野用にと組み合わせ,低コストで大型ディスプレイを視聴しているのと近い状況を構築する手法を提案している\cite{shamo9}.
また視覚的刺激の提示の有 効性はディスプレイ上への提示に限らず,ulumiRoom は,ゲームプレイ時のディスプレイ周辺の壁や床に,
プロジェクタからゲームに対応したコンテンツの投影により,臨場感や迫力の増強する手法を実現し \cite{shamo10},低解像度LEDマトリクスを使用し、
適切なオプティカルフローを周辺視に提示をおこない,速度 感を提示する手法を実現している \cite{shamo11}.ディスプレイの四周にLEDアレイを配置し,
ユーザの周辺視野へ動きの提示により,スピード感を増強させるシステムなどを実現し,LED点滅パターンの制御をおこない,
動きを提示しユーザの感じるスピード感の増強を狙った研究である\cite{shamo12}.
また,自宅で4DX体験が可能な映像に合わせて温風や冷風を噴射するVortxは,
映像中に炎が吹き出すシーンで温風が噴射され映像の炎を熱いと感じれ,
空を飛ぶシーンでは冷風を噴射させ空気の抵抗/圧力を感じ取れるデバイスである(図\ref{vortx}).
Vortxの再生装置としてWindowsPCが必要であるが特別な装置やアプリケーションは不要で,USBでPCと接続するのみとなっている.
画面に再生されているYouTubeやNetflix,ブルーレイドライブで再生される映像,
ネット対戦ゲーム等々,PCのモニターに表示されている映像に対して
特許取得済みEnvironmentalExperience(EX)アルゴリズムによってリアルタイムで分析/認識して即座に対応が可能である\cite{vortx}.
本研究は,出力装置を用いず周辺視野を刺激し,動画視聴に新たな効果を加える. 

\begin{figure}[H]
  \centering
  \includegraphics[width=15cm]{images/chapter2/vortx.png}
  \caption{4D体験可能なデバイス:Vortx}
  \label{vortx}
\end{figure}