\thispagestyle{myheadings}
\chapter{関連研究}
 本章では本研究に関連するコンテンツについての研究や生体データを用いた研究,重畳提示手法の研究の3つのテーマについて紹介する.本研究と比較し関連性や新規性を述べる.

\section{コンテンツにおける研究}
 コンテンツにはたくさんの種類がある.アナログコンテンツである本や新聞紙などがあるが,本や新聞紙など紙で読む方が記憶に残る研究がある.しかし,スマートフォンの普及で紙で読む機会が減った.本研究のシステムを使えば,紙で本や新聞紙を読む機会も増えると考える.
\section{生体データを用いた研究}
 生体データを使った研究はいくつかある.その中で生体データを使い人の感情をどれだけ正確に抜き出せるのか,また生体データを基にした研究について本研究における生体データの取扱について比較し述べる.
\subsection{生体情報で何がわかるかの研究}
 生体データを取得したときに得られる情報について述べる.我々の研究では生体データを用いてコンテンツに重畳するため生体データでコンテンツに対して視聴者の反応を取る必要がある.生体データを使い人の感情を読み取る研究がある[1].これらにより生体データを使い感情が読み取れる有効性が示されていた。また、生体データを用いて被験者に学習支援をする研究もある[3][6].生体データを使用し正確に支援ができるからである.そのため本研究では生体データを使用した.
\subsection{生体情報を基にした研究}
 また,生体データを用いて被験者に学習支援をする研究もある[][].生体データを使うことでより正確に支援ができるからである.そのため本研究では生体データを使うこととした.

\section{重畳提示についての研究}
コンテンツにエフェクトを重畳して面白さを増幅する研究はこれまで多くなされてきた。視線検出装置を用いてユーザがディスプレイ上の画像において任意の点に注目した際に,奥行きにフォーカスされた画像を映像に随時反映し奥行き感を強化する手法が提案されている[7][8].しかし、提案システムはあらかじめ用意した最大 256枚からなる画像を利用しており,手軽にコンテンツの視聴体験を拡張,増幅できるとは言い難く,また視線位置に応じて仮想世界のぼかし具合を変化させ,面白さや奥行き感を増幅している[9].そのため本研究では重畳提示手法を使用した.

