\thispagestyle{myheadings}
\chapter{関連研究}


\section{生体データを用いた研究}
 生体データを取得したときに得られる情報について述べる。我々の研究では生体データを用いてコンテンツに重畳するため生体データでコンテンツに対して視聴者の反応を取る必要がある。生体データを使い人の感情を読み取る研究がある[1].これらにより生体データを使い感情が読み取れる有効性が示されていた。また、生体データを用いて被験者に学習支援をする研究もある[3][6].生体データを使うことでより正確に支援ができるからである。そのため本研究では生体データを使うこととした。


\section{重畳提示についての研究}
 コンテンツにエフェクトを重畳して面白さを増幅する研究はこれまで多くなされてきた。岡谷ら[6]の研究では,視線検出装置を用いてユーザがディスプレイ上の画像において任意の点に注目した際に,奥行きにフォーカスされた画像を映像に随時反映し奥行き感を強化する手法が提案されている.しかし、岡谷らの提案システムはあらかじめ用意した最大 256枚からなる画像を利用しており,手軽にコンテンツの視聴体験を拡張,増幅できるとは言い難い.また,Hillaire ら[7]は視線位置に応じて仮想世界のぼかし具合を変化させ,面白さや奥行き感を増幅している.しかし,仮想世界のみを対象としているものであり,コンテンツ内のレンダリングに使用されているカメラのデータを用いる必要があるため汎用性に欠ける.


