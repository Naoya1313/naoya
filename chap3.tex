\chapter{システム全体像}
\thispagestyle{myheadings}

\section{全体の流れ}

\subsection{スマートウォッチを使い心拍数を測定}

\subsection{集めたデータ処理}

\subsection{エフェクト表示}

\section{データの処理}

\subsection{生体データの観察}

\subsection{データの処理方法}

\section{エフェクト表示}



\subsection{electronでの表示方法}


エフェクト表示方法として,カテゴリやレベルごとにエフェクトを用意し,動画として重畳表示しているが,レベル分けを行ったためエフェクトをダイナミックに生成できず,細かな心拍数変化を表示できていない.


\subsection{エフェクトの種類}

エフェクトの種類としてAction, Anime, Horrorの3種類に分け見る映画のジャンルを選択可能にした.選択できるようにした理由として,視聴ジャンルによって同じエフェクト効果を表示した際に,エフェクトが映像視聴の際に邪魔になってしまう点が挙げられる.
エフェクトの目的として,Actionのエフェクトは迫力・緊迫感を与え,Animeのエフェクトは面白さを与える.Horrorのエフェクトは恐怖を軽減させる目的として制作した.心拍データの処理の際に1から3までのレベル分けの結果でエフェクトが変更される.1から3に分けた際に1の結果が出力された時間はエフェクトを透過し映像視聴の邪魔にならないように制作し,2と3の結果が出力されている時のみに2段階に分けエフェクトが表示されるように制作した.