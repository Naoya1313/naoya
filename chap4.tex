\thispagestyle{myheadings}
\chapter{評価実験}

\section{実験}
\subsection{実際に集めたデータを観察}
\subsection{アンケート実施}
エフェクト選択後の視聴画面に対し10人のデータを収集し,評価方法としてエフェクト制作の目的を5段階評価でどれほど感じたかで集めた

\section{考察}
\subsection{データの観察を基に閾値の設定}
\subsection{重畳するエフェクト}
データを処理した結果として,Actionのエフェクトではキングコングを視聴し5の評価6件4の評価を6件と恐怖感や緊迫感を感じた意見があった.しかしエフェクトの色が赤色のみに固定されている.エフェクトが急に激しくなる意見に対応ができていなかった.改善策としてエフェクトの色を選択できるように変更し,エフェクト表示が変わる時にフェードアウトを追加する.AnimeのエフェクトではSPY×FAMILYを試聴し5の評価3件4の評価6件3の評価1件と面白さを与えるのが低かった.エフェクトの線が多く邪魔になった.アニメの中でもSFやコメディーなどに合わない意見に対応ができていなかった.改善策として効果線の太さを細くし変更,アニメのジャンルの中でもエフェクト選択画面を追加する.Horrorのエフェクトでは嘘喰いを視聴し4の評価2件4の評価6件3の評価2件と恐怖を軽減するのが低かった.画面全体が急に見づらくなった.恐怖を軽減するのに対応ができていなかった.改善策としてエフェクトのぼかしの不透明度を上げ,ぼかす範囲を狭める.

\section{検証}
\subsection{データの観察を基に閾値の設定}
\subsection{重畳するエフェクト}