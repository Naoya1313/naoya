\thispagestyle{myheadings}
\chapter{評価実験}

収集した心拍データの処理が適切なのか、また表示 したエフェクトが視聴者にはどのように感じるのかを 評価するため実験を行う.データ収集方法として安静 にした状態で腕にスマートウォッチを取り付け、映画 を見る1分前から心拍数の取得を始め映像が終わるまでを計測した.エフェクトを重畳した映画の視聴後のアンケートも行った. 

\section{生体データの処理方法}
データを処理する際の閾値や設定が正確に処理でき ていたか評価する.3 本の映画に対し各 3 人のデータを 収集した.映画は 007/ノー・タイム・トゥ・ダイ、キングコング/髑髏島の巨神,嘘喰いの 3 本にした.集めたデータを処理した結果、盛り上がりのシーンで心拍 数が上昇しておりその部分が抜き出せていた.エフェ クトを表示する時間 from から to も閾値を超えた瞬間から下がるまでを記録できていた。しかし人によっては心拍数が常に一定の人や変化が 激しい人など様々なためその全てに対応ができていな かった.原因として始めの一分間を安静時の心拍数と して決定するためその後の変化に対応できないからだと推測する.また、処理したそれぞれのデータを一つ のデータに追加していくだけなので、姿勢を変えたなどのノイズもエフェクト表示に使われてしまった.対 策として様々な心拍数の変化に対応するため、平均を 常に更新しそれに応じて閾値も変動させ、データを一 つにまとめる際には複数人が反応を示している箇所だけを抜き出しまとめれば解決すると考える. 

\section{重畳提示の手法}
エフェクト選択後の視聴画面に対し10人のデータを収集し,評価方法としてエフェクト制作の目的を5段階評価でどれほど感じたかで集めた.
データを処理した結果として,Actionのエフェクトではキングコングを視聴し5の評価6件4の評価を6件と恐怖感や緊迫感を感じた意見があった.しかしエフェクトの色が赤色のみに固定されている.エフェクトが急に激しくなる意見に対応ができていなかった.改善策としてエフェクトの色を選択できるように変更し,エフェクト表示が変わる時にフェードアウトを追加する.AnimeのエフェクトではSPY×FAMILYを試聴し5の評価3件4の評価6件3の評価1件と面白さを与えるのが低かった.エフェクトの線が多く邪魔になった.アニメの中でもSFやコメディーなどに合わない意見に対応ができていなかった.改善策として効果線の太さを細くし変更,アニメのジャンルの中でもエフェクト選択画面を追加する.Horrorのエフェクトでは嘘喰いを視聴し4の評価2件4の評価6件3の評価2件と恐怖を軽減するのが低かった.画面全体が急に見づらくなった.恐怖を軽減するのに対応ができていなかった.改善策としてエフェクトのぼかしの不透明度を上げ,ぼかす範囲を狭める.

