\chapter{おわりに}

\section{まとめ}
今回生体データを用いてコンテンツに重畳し、コンテンツをより盛り上げるシステムを提案した。コンテンツ視聴時の生体データを収集し盛り上がっている箇所を推定、抽出しコンテンツに重畳できるデータに変換する。そのデータをもとにコンテンツへエフェクトなどの重畳をする。本研究では生体データとして心拍数を利用し、コンテンツを映画にした。映画を視聴している時の心拍数を計測し、心拍数が上昇している箇所を抜き出す。抜き出したデータを基にエフェクトを映画画面に重畳する。
評価実験を通して心拍数上昇箇所の抽出処理における課題点、コンテンツへ重畳するエフェクトの問題点が明らかとなった。心拍数上昇箇所の抽出処理において、心拍数の変動はさまざまであるためそれに対応ができていなかった。改善策として心拍数上昇箇所を抽出するときに安静時の心拍数の設定で対応できると考察した。


\section{今後の課題}
今後の課題を心拍数上昇箇所の抽出処理、コンテンツへのエフェクト表示の双方において述べる。まずデータ処理の課題として挙げられるのは、心拍数上昇箇所を抽出するときに体を動かしただけやくしゃみをしたなどの映画の盛り上がりとは関係ないところでも心拍数が上昇しており、そのノイズを除去するのができていなかった。また人によって心拍数の変動が様々であるため全ての人に対応できる処理ではなかった。まず前者に対応するため心拍数上昇箇所の抽出処理をするときに最初の一分間を安静時の心拍数としたが、安静時の心拍数を常に更新することで解決すると考える。心拍数が盛り上がる前後では心拍数は比較的落ちつていたため、この落ち着いている場所で安静時の心拍数の平均を取る。この平均を使い閾値を設定し、さまざまな心拍数のパターンに対応する。後者の問題として、心拍数上昇箇所の抽出処理した後のデータをエフェクトを重畳するためのデータにするため一つのデータに足していくだけであった。そこを複数人が同じ箇所で反応していたところのみ最後の一つのデータに追加していくことでノイズを除去したりするのを対策可能である。少ない人数間での比較だったため将来的にはより多くのデータを集める必要がある.
エフェクトの種類を追加しジャンルごとの選択画面の追加が必要である. 
