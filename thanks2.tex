\chapter*{参考文献}
\addcontentsline{toc}{chapter}{\protect\numberline {} 参考文献}

[1] 高橋和彦, “マルチモーダル生体信号情報による感 情認識に関する一考察”, 人間工学, Vol. 41, No, 4, pp. 248-253, 2005.

[2] 高野 研一, 長坂 彰彦, 吉野 賢治, “生体情報を利用した作業者の心身状態評価法の現状と動向”, 産業医学, Vol. 34, No, 2, pp. 95-115, 1992.

[3] 松居 辰則, “生体情報を用いた学習者の心的状態推定と学習支援の試み”, 教育システム情報学会誌, Vol. 36, No, 2, pp. 76-83, 2019.

[4] 松居辰則, 田和辻可昌, “深層ニューラルネットワーク を用いた学習者の生体情報からの心的状態推定モデル における中間層の可視化の試み”, 教育工学, Vol. 118, No, 214, pp. 31-36, 2018.

[5] 松居辰則,宇野達朗,田和辻可昌, “心的状態の時間遅れと持続モデルを考慮した生体情報からの学習者の心的状態推定の試み”, 第32回人工知能学会全国大会, 4H1-OS-9a-05, 2018.

[6] 中山 実, 清水 康敬, “生体情報による学習活動の評価”, 日本教育工学会論文誌, Vol. 24, No, 1, pp. 15-23, 2000.

[7] 松井啓司ら, ”周辺視へのエフェクト提示による動 画の視聴体験拡張”, エンタテイメントコンピューテ ィングシンポジウム, Vol.2015, pp.543-550, 2015. 

[8]岡谷貴之,石澤品,出口光一郎:被军界深度ぼけの提示により 奥行感を強化する注視反応型ディスプレイ,電子情報通信学会論文誌 Vo1.J92-D No.8 ,2009.

[9]S. Hillaire, A. Lécuyer, R. Cozot, G. Casiez: Using an Eye-Tracking System to Improve Camera Motions and Depth-of-Field Blur Effects in Virtual Environments, IEEE Virtual Reality Conference, pp.47-50,2008.
